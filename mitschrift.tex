% Dokumententyp (Schriftgröße, Papiergröße, Formelausrichtung)
\documentclass[pdftex,12pt,a4paper,fleqn]{scrartcl}

% Ein- und Ausgabekodierung
\usepackage[utf8]{inputenc}
\usepackage[T1]{fontenc}

% Sprache
\usepackage[ngerman]{babel}
\usepackage[babel,german=quotes]{csquotes}

% Schriftart
\usepackage{lmodern}

% Euro-Symbol
\usepackage{eurosym}

% Tabellen
\usepackage{longtable}
\usepackage{array}

% PDF-Seiten Landscape
\usepackage{pdflscape}

% Schlussregeln
\usepackage{proof}

% Listen
\usepackage{enumitem}

% Formelsatz
\usepackage{amsmath}
\usepackage{amsfonts}
\usepackage{amssymb}

% Seiten-Geometrie
\usepackage[paper=a4paper,margin=2.5cm]{geometry}

% Zitate
\usepackage[backend=biber,style=alphabetic]{biblatex}

% Hurenkinder- und Schusterjungenregelung
\clubpenalty = 10000
\widowpenalty = 10000
\displaywidowpenalty = 10000

% Variablen
\newcommand{\doctitle}{Mathematik für Informatiker 1}
\newcommand{\docsubtitle}{}
\newcommand{\docauthor}{Marvin Homann}
\newcommand{\docdate}{21. Oktober 2015 ff.}

% PDF-Links
\usepackage[ngerman,pdfauthor={\docauthor},
pdftitle={\doctitle},
colorlinks=true,linkcolor=black,citecolor=black,urlcolor=black,
filecolor=black]{hyperref}

% Titelseite
\title{\doctitle}
\subtitle{\docsubtitle}
\author{\docauthor}
\date{\docdate}

% Zeilenabstand
\usepackage{setspace}

% Kopf- und Fußzeilen
\usepackage[automark]{scrpage2}
\ihead[]{}
\ohead[]{}
\ifoot[]{\docauthor}
\cfoot[]{}
\ofoot[]{\thepage}
\pagestyle{scrheadings}

% Fußnoten
\usepackage[hang]{footmisc}
\setlength{\footnotemargin}{3mm}

% Todonotes
\usepackage{todonotes}

% Boxen
\usepackage[framemethod=tikz]{mdframed}

% Grafiken und Fotos
\usepackage{tikz}
\usepackage{tikz-qtree}
\usepackage{circuitikz}
\usetikzlibrary{decorations}
\usepackage{graphicx}
\usepackage{wrapfig}
\usepackage{caption}
\usepackage{subcaption}

% Farben
\usepackage{color}

% Floating
\usepackage{float}

% Formatierung der Paragraphen
\parindent 0pt
\parskip 6pt

% Euro- und Gradzeichen (Unicode)
\DeclareUnicodeCharacter{B0}{$^{\circ}$}
\DeclareUnicodeCharacter{20AC}{\euro}
 
\begin{document}
 
\maketitle
\tableofcontents
\newpage

\section{Organisation} % (fold)
\label{sec:organisation}

% section organisation (end)

\subsection{Prof} % (fold)
\label{sub:prof}
\begin{itemize}
	\item Prof. Hanna Markwig
	\begin{itemize}
	 	\item Raum: E2.4, 301
	 	\item Tel: 57422
	 	\item Mail: hannah@math.uni-sb.de
	 	\item Web: \href{https://www.math.uni-sb.de/wiki/doku.php?id=ag-seite:ag-markwig:teaching:eaz}{Mathe für Informatiker I}
	 	\item Anmeldung für die Übungsgruppen bis Freitag 23.10.2015 10 Uhr
	 	\item \href{https://forum.math.uni-sb.de/viewforum.php?f=53}{Forum:} Passwort: Leibniz
	 \end{itemize} 
	 \item Assistent Christian Jürgens
	 \begin{itemize}
	 	\item Raum: E2.4, 303
	 	\item Tel: 3206
	 	\item Mail: juergens@math.uni-sb.de
	 \end{itemize}
\end{itemize}
% subsection prof (end)

\subsection{Literatur} % (fold)
\label{sub:literatur}
\begin{itemize}
	\item Forster, Analysis 1
	\item Kreußler, Pfister, Mathematik für informatiker
	\item Schreyer, Mathematik für Informatiker
	\item Gathmann, Grundlagen Mathematik
	\item Markwig, Grundlagen Informatik
	\item Heuser, Analysis 1
	\item Manfred Lehn, Wie bearbeite ich einen Übungszettel?
\end{itemize}
% subsection literatur (end)

\subsection{Übungszettel} % (fold)
\label{sub:_bungszettel}
\begin{itemize}
	\item Jeden Freitag gibt es einen Übungszettel
	\item Eine Woche Zeit, diesen Zettel zu bearbeiten
	\item Übungsgruppen beginnen ab 2.11.2015
\end{itemize}
% subsection _bungszettel (end)

\subsection{Klausuren} % (fold)
\label{sub:klausuren}
\begin{itemize}
	\item Hauptklausur 19.2.2016
	\item Nachklausur 6.4.2016
\end{itemize}
% subsection klausuren (end)

\newpage

\section{Vorlesung 21.10.2015} % (fold)
\label{sec:vorlesung_21_10_2015}

\subsection{Übersicht} % (fold)
\label{sub:_bersicht}
\begin{itemize}
	\item Reelle Zahlen und Funktionen von reellen Zahlen
	\item Grenzwerte
	\item Stetigkeit
	\item Differenzierbarkeit
	\item Integralrechnung
\end{itemize}
% subsection _bersicht (end)

\subsection{Grundlagen} % (fold)
\label{sub:grundlagen}
\subsubsection{Aussagenlogik} % (fold)
\label{ssub:aussagenlogik}
Aussagen sind Sätze, für die man entscheiden kann, ob sie wahr oder falsch sind.
Man kann ihnen einen Wahrheistwert zuweisen.

\textbf{Beispiele:} 
\begin{itemize}
	\item Das Saarland ist ein Bundesland der BRD ist eine wahre Aussage
	\item Das Saarland ist das größte Bundesland der BRD ist eine falsche Aussage
	\item Knödel schmecken nicht ist keine Aussage
	\item Mach das Licht an ist  keine Aussage
	\item $1+1=2$ ist eine wahre Aussage
	\item $1+1=3$ ist eine falsche Aussage
	\item $1+1$ ist keine Aussage
\end{itemize}

\textbf{Ziel:} aus bekannten wahren Aussagen neue Aussagen als wahr nachzuweisen.

\textbf{Beispiele:}
\begin{itemize}
	\item $A:$ Der Bundespärsident ist stets mindestens 40 Jahre alt ist wahr
	\begin{itemize}
		\item Das Grundgesetz zur Wahl des Bundespräsidenten: 

		$B:$ Wählbar ist jeder Deutsche, der das Wahlrecht zum Bundestag besitzt und das 40te Lebensjahr beendet hat
		\item $B \Rightarrow A$ oder Aus B folgt A
	\end{itemize}
\end{itemize}

Aussagen, die als wahr festgelegt werden, heißen \textbf{Axiome}.

\textbf{Negation:}
$\neg A$ Der Bundespräsident ist nicht stets mindestens 40 Jahre alt

\textbf{Konjunktion (und):} $\land$ 

$C:$ Wählbar ist jeder Deutsche, der das Wahlrecht zum Bundestag besitzt

$D:$ Wählbar ist jeder Deutsche, der das 40. Lebensjahr vollendet hat.

$C \land D$ stimmt mit B überein

$C \land D$ ist nur dann wahr, wenn C und D beide wahr sind.

\begin{tabular}{@{ }c@{ }@{ }c | c@{}@{ }c@{ }@{ }c@{ }@{ }c@{ }@{}c@{ }}
C & D & ( & C & $\land$ & D & )\\
\hline 
$\top$ & $\top$ &  & $\top$ & \textcolor{red}{$\top$} & $\top$ & \\
$\top$ & $\bot$ &  & $\top$ & \textcolor{red}{$\bot$} & $\bot$ & \\
$\bot$ & $\top$ &  & $\bot$ & \textcolor{red}{$\bot$} & $\top$ & \\
$\bot$ & $\bot$ &  & $\bot$ & \textcolor{red}{$\bot$} & $\bot$ & \\
\end{tabular}

\textbf{Disjunktion (oder):}
$E \lor F$ ist wahr, wenn E oder F wahr sind.

%NOTE: requires \usepackage{color}
\begin{tabular}{@{ }c@{ }@{ }c | c@{}@{ }c@{ }@{ }c@{ }@{ }c@{ }@{}c@{ }}
E & F & ( & E & $\lor$ & F & )\\
\hline 
$\top$ & $\top$ &  & $\top$ & \textcolor{red}{$\top$} & $\top$ & \\
$\top$ & $\bot$ &  & $\top$ & \textcolor{red}{$\top$} & $\bot$ & \\
$\bot$ & $\top$ &  & $\bot$ & \textcolor{red}{$\top$} & $\top$ & \\
$\bot$ & $\bot$ &  & $\bot$ & \textcolor{red}{$\bot$} & $\bot$ & \\
\end{tabular}

\textbf{Äquivalenz (gleich):}
$B \Leftrightarrow (C \land D)$ B ist äquivalent zu $C \land D$, wenn beide den selben Wahrheitswert haben.

%NOTE: requires \usepackage{color}
\begin{tabular}{@{ }c@{ }@{ }c@{ }@{ }c | c@{}@{ }c@{ }@{ }c@{ }@{}c@{}@{ }c@{ }@{ }c@{ }@{ }c@{ }@{}c@{}@{}c@{ }}
B & C & D & ( & B & $\Leftrightarrow$ & ( & C & $\land$ & D & ) & )\\
\hline 
$\top$ & $\top$ & $\top$ &  & $\top$ & \textcolor{red}{$\top$} &  & $\top$ & $\top$ & $\top$ &  & \\
$\top$ & $\top$ & $\bot$ &  & $\top$ & \textcolor{red}{$\bot$} &  & $\top$ & $\bot$ & $\bot$ &  & \\
$\top$ & $\bot$ & $\top$ &  & $\top$ & \textcolor{red}{$\bot$} &  & $\bot$ & $\bot$ & $\top$ &  & \\
$\top$ & $\bot$ & $\bot$ &  & $\top$ & \textcolor{red}{$\bot$} &  & $\bot$ & $\bot$ & $\bot$ &  & \\
$\bot$ & $\top$ & $\top$ &  & $\bot$ & \textcolor{red}{$\bot$} &  & $\top$ & $\top$ & $\top$ &  & \\
$\bot$ & $\top$ & $\bot$ &  & $\bot$ & \textcolor{red}{$\top$} &  & $\top$ & $\bot$ & $\bot$ &  & \\
$\bot$ & $\bot$ & $\top$ &  & $\bot$ & \textcolor{red}{$\top$} &  & $\bot$ & $\bot$ & $\top$ &  & \\
$\bot$ & $\bot$ & $\bot$ &  & $\bot$ & \textcolor{red}{$\top$} &  & $\bot$ & $\bot$ & $\bot$ &  & \\
\end{tabular}

\textbf{Implikation (folgt):}
$E \Rightarrow F$ aus der Wahrheit von E können wir Rückschlüsse auf die von F ziehen.

%NOTE: requires \usepackage{color}
\begin{tabular}{@{ }c@{ }@{ }c | c@{}@{ }c@{ }@{ }c@{ }@{ }c@{ }@{}c@{ }}
E & F & ( & E & $\Rightarrow$ & F & )\\
\hline 
$\top$ & $\top$ &  & $\top$ & \textcolor{red}{$\top$} & $\top$ & \\
$\top$ & $\bot$ &  & $\top$ & \textcolor{red}{$\bot$} & $\bot$ & \\
$\bot$ & $\top$ &  & $\bot$ & \textcolor{red}{$\top$} & $\top$ & \\
$\bot$ & $\bot$ &  & $\bot$ & \textcolor{red}{$\top$} & $\bot$ & \\
\end{tabular}

Wir zeigen: $E \Rightarrow F$ und $\neg E \lor F$ sind äquivalent.

Vergleich der Wahrheitstabellen:

%NOTE: requires \usepackage{color}
\begin{tabular}{@{ }c@{ }@{ }c | c@{}@{}c@{}@{ }c@{ }@{ }c@{ }@{ }c@{ }@{}c@{}@{ }c@{ }@{}c@{}@{ }c@{ }@{ }c@{ }@{ }c@{ }@{ }c@{ }@{}c@{}@{}c@{ }}
E & F & ( & ( & E & $\Rightarrow$ & F & ) & $\Leftrightarrow$ & ( & $\neg$ & E & $\lor$ & F & ) & )\\
\hline 
$\top$ & $\top$ &  &  & $\top$ & $\top$ & $\top$ &  & \textcolor{red}{$\top$} &  & $\bot$ & $\top$ & $\top$ & $\top$ &  & \\
$\top$ & $\bot$ &  &  & $\top$ & $\bot$ & $\bot$ &  & \textcolor{red}{$\top$} &  & $\bot$ & $\top$ & $\bot$ & $\bot$ &  & \\
$\bot$ & $\top$ &  &  & $\bot$ & $\top$ & $\top$ &  & \textcolor{red}{$\top$} &  & $\top$ & $\bot$ & $\top$ & $\top$ &  & \\
$\bot$ & $\bot$ &  &  & $\bot$ & $\top$ & $\bot$ &  & \textcolor{red}{$\top$} &  & $\top$ & $\bot$ & $\top$ & $\bot$ &  & \\
\end{tabular}

Warum wird in $E \Rightarrow F$ der Wahrheitswert $\top$ zugeordnet, wenn E falsch ist.

\textbf{Beispiel:}

$x^2-2x=-1$

$x^2-2x+1=0$

$x^2-2x=-1 \Rightarrow x^2-2x+1=0$

$1=0 \Rightarrow 1+1=0+1$

$1=0$ ist falsch, aber der Schluss ist richtig (auf beiden Seiten 1 addieren, ändert nichts)

\textbf{Kontraposition:}
Es gilt: $X \Rightarrow Y$ und $\neg Y \Rightarrow \neg X$ sind äquivalent.

%NOTE: requires \usepackage{color}
\begin{tabular}{@{ }c@{ }@{ }c | c@{}@{}c@{}@{ }c@{ }@{ }c@{ }@{ }c@{ }@{}c@{}@{ }c@{ }@{}c@{}@{ }c@{ }@{ }c@{ }@{ }c@{ }@{ }c@{ }@{ }c@{ }@{}c@{}@{}c@{ }}
X & Y & ( & ( & X & $\Rightarrow$ & Y & ) & $\Leftrightarrow$ & ( & $\neg$ & Y & $\Rightarrow$ & $\neg$ & X & ) & )\\
\hline 
$\top$ & $\top$ &  &  & $\top$ & $\top$ & $\top$ &  & \textcolor{red}{$\top$} &  & $\bot$ & $\top$ & $\top$ & $\bot$ & $\top$ &  & \\
$\top$ & $\bot$ &  &  & $\top$ & $\bot$ & $\bot$ &  & \textcolor{red}{$\top$} &  & $\top$ & $\bot$ & $\bot$ & $\bot$ & $\top$ &  & \\
$\bot$ & $\top$ &  &  & $\bot$ & $\top$ & $\top$ &  & \textcolor{red}{$\top$} &  & $\bot$ & $\top$ & $\top$ & $\top$ & $\bot$ &  & \\
$\bot$ & $\bot$ &  &  & $\bot$ & $\top$ & $\bot$ &  & \textcolor{red}{$\top$} &  & $\top$ & $\bot$ & $\top$ & $\top$ & $\bot$ &  & \\
\end{tabular}

% subsubsection aussagenlogik (end)

% subsection grundlagen (end)

% section vorlesung_21_10_2015 (end)

\newpage
\section{Vorlesung 23.10.2015} % (fold)
\label{sec:vorlesung_23_10_2015}

\subsection{Aussageformen} % (fold)
\label{sub:aussageformen}

Eine Aussageform ist eine Äußerung, die eine oder mehrere Variablen enthält und zu einer AUssage wird, wenn man Werte für die Variablen einsetzt.

\textbf{Beispiel: } $ a > b$ for $a = 42$ und $b = 3$ ist eine wahre Aussage.

\subsubsection{Quantoren} % (fold)
\label{ssub:quantoren}

Quantoren treffen eine Aussage über alle möglichen Werte.

$\forall:$ Diese Aussage ist wahr, wenn sie für alle Werte gilt.

$\exists:$ Diese Aussage ist wahr, wenn sie für mindestens einen Wert gilt.

$\exists!:$ Diese Aussage ist wahr, wenn sie für genau einen Wert gilt.

$\not\exists:$ Diese Aussage ist wahr, wenn sie für keinen Wert gilt.

% subsubsection quantoren (end)

\subsubsection{Zahlen} % (fold)
\label{ssub:zahlen}

Kurzschreibsymbole für häufig verwendete Zahlenmengen.

$\mathbb{N}:$ Die natürlichen zahlen inklusive der Null. $\{0,1,2,3...\}$

$\mathbb{Z}:$ Die ganzen Zahlen, positiv und negativ. $\{...,-2,-1,0,1,2,...\}$

$\mathbb{Q}:$ Die rationalen Zahlen, Brüche aus $frac{\mathbb{Z}}{\mathbb{Z}}$

$\mathbb{R}:$ Die reellen Zahlen, beliebige Kommazahlen $\{\pi,3.45,1,-4.5656...,...\}$

% subsubsection zahlen (end)

% subsection aussageformen (end)

\subsection{Mengen} % (fold)
\label{sub:mengen}

\subsubsection{Charakterisierung von Cantor} % (fold)
\label{ssub:charakterisierung_von_cantor}

Eine Menge ist eine Zusammenfassung von bestimmten wohlunterschiedenen Objekten unserer Anschauung oder unseres Denkens zu einem Ganzen. Die in einer Menge zusammengefassten Objekte heißen Elemente.

% subsubsection charakterisierung_von_cantor (end)

\subsubsection{Notation} % (fold)
\label{ssub:notation}

Eine Auflistung von Elementen zu einer Menge wird in geschweiften Klammern getan, Reihenfolge und Mehrfachnennung spielen dabei keine Rolle. $\{1,2,3\}$

Man kann auch eine Eigenschaft voranschreiben: $\{x | x \in \mathbb{N} < 6\} = \{0,1,2,3,4,5\}$

Sei M eine Menge, $x \in M $ heißt, dass x ein Element von M ist. $x \notin M$ heißt, dass x kein Element von M ist.

Seien M und N Mengen: $M \subset N$ bedeutet $(c \in M \Rightarrow x \in N)$ wir sagen M ist eine Teilmenge von N. $M \subsetneq N$ bedeutet $M \subset N \land M \neq N$ wir sagen M ist eine echte Teilmenge von N.

Seien M, N und P Mengen:

$M \cap N$ definiert als $\{x | x \in M \land x \in N\}$ ist der Schnitt, wir sagen M geschnitten N.

$M \cup N$ definiert als $\{x | x \in M \lor x \in N$ ist die Vereinigung, wir sagen M vereinigt mit N.

$M \setminus N$ definiert als $\{x | x \in M \land x \notin N\} $ wir sagen M ohne N. 

M und N heißen disjunkt, wenn gilt $M \cap N = \emptyset$

$P = M \mathbin{\dot{\cup}} N$ ist die disjunkte Vereinigung $P = M \cup N$ mit M und N disjunkt.

$M \times N$ definiert als $\{(x,y) | x \in M \land y \in N\}$ ist das kartesische Produkt von M und N. Das Tupel $(x,y)$ ist ein geordnetes Paar, auch im Fall $M=N$. 

Es gilt weiterhin $(x,y)=(x',y') \Leftrightarrow x=x' \land y=y'$

Für den Fall von $M = N$ schreibt man auch $M^2 = M \times M$.

\textbf{Beispiel:} $M=\{1,2\}, N=\{0,2\}$

$M \times N = \{\{1,0\},\{1,2\},\{2,0\},\{2,2\}\}$

\textbf{Bemerkung:} Um zu zeigen, dass zwei Mengen M und N gleich sind, zeigt man $M \subset N$ und $N \subset M$.

% subsubsection notation (end)

% subsection mengen (end)

\subsection{Beweisprinzipien} % (fold)
\label{sub:beweisprinzipien}

\subsubsection{Widerspruchsbeweis} % (fold)
\label{ssub:widerspruchsbeweis}

In einem Widerspruchsbeweis führt man die Annahme des Gegenteils zum Widerspruch.

\textbf{Definition:} Eine Primzahl ist eine natürliche Zahl $p \in \mathbb{N}: p > 1$, welche nur durch 1 und p ohne Rest teilbar ist.

\textbf{Satz von Euklid:} Es gibt unendlich viele Primzahlen.

\textbf{Beweis:} Angenommen, es gäbe nur endlich viele Primzahlen $p_1$ bis $p_n$. $q$ definiert als $p_1+...+p_n+1$. Nun ist q durch kein p teilbar, da immer der Rest 1 bleibt. Also ist q selbst oder ein Primteiler von q eine Primzahl, die in der Liste $p_1+...+p_n$ nicht vorkommt.

% subsubsection widerspruchsbeweis (end)

\subsubsection{Vollständige Induktion} % (fold)
\label{ssub:vollst_ndige_induktion}

In der vollständigen Induktion zeigt man von einem wahren Start und einem wahren Schritt, dass die Behauptung für alle Elemente von Start in Richtung Schritt gilt.

Sei $A(n)$ eine Aussageform mit zulässigen Werten $n \in \mathbb{N}$. Ist die Aussageform $A(0)$ wahr und $A(n) \Rightarrow A(n+1)$, so ist $A(n)$ wahr für alle $n \in \mathbb{N}$.

$A(0)$ ist der \underline{Induktionsanfang}.

$A(n)$ ist die \underline{Induktionsannahme}.

$A(n) \Rightarrow A(n+1)$ ist der \underline{Induktionsschritt}

\textbf{Beispiel:} Die Zahl $n^3-n$ ist für alle $n \in \mathbb{N}$ durch 6 ohne Rest teilbar.

\textbf{Beweis:} $A(n): \exists k \in N: n^3-n=6k$

Induktionsannahme $n = 0$: $0^3-0=6*0$ ist mit $k=0$ wahr.

Induktionsannahme: $\exists k \in N: n^3-n = 6k$

Induktionsschritt $n \Rightarrow n+1$:  

Entweder n oder n+1 ist eine gerade Zahl, daher ist das Produkt $n(n+1)$ gerade.

Daher gilt $\exists l \in \mathbb{N}: n(n+1)=2 \cdot l \cdot k$

Damit gilt $(n+1)^3-(n-1) = (n^3 + 3n^2 + 3n +1) - (n + 1) = (n^3 -n) + 3(n^2+n)=n^3-n+3(n+1)\cdot n = (Mit\ Induktionsannahme) 6k+3(n+1)\cdot n = 6k + 3 \cdot 2 \cdot l = 6(k+l)$

Womit $A(n+1)$ wahr ist.

\textbf{Variander der vollständigen Induktion}
\begin{itemize}
    \item Statt bei 0 kann man auch bei einem beliebigen $n_0$ starten und zeigen, dass $A(n) $ für alle $n \geq n_0$ wahr ist.
    \item Man kann den Induktionsschritt in die andere Richtung führen.
    \item Man kann die Induktion für endlich viele $n \in N$ zeigen.
    \item Man kann den Induktionsschritt größer machen und so zum Beispiel für alle gerade zahlen zeigen, dass $A(n)$ gilt. 
\end{itemize}

% subsubsection vollst_ndige_induktion (end)

% subsection beweisprinzipien (end)

\subsection{Abbildungen} % (fold)
\label{sub:abbildungen}

\textbf{Definition:} Seien M und N Mengen. Eine Abbildung oder Funktion f von M nach N ist eine \underline{eindeutige Zuordnung}, die \underline{jedem Element} $x \in M$ \underline{genau ein Element} $f(x) \in N$ zuweist. \underline{M heißt Definitionsbereich}, \underline{N heißt Zielbereich oder Wertebereich.}

\textbf{Notation} $f: M \rightarrow N$

Dabei ist es egal, wie die Zuordnung angegeben wird.

Sei M eine Menge: $id_M$ ist definiert als $ M \rightarrow M: x \mapsto x$ die Identität von M.

\textbf{Beispiele für Funktionen:}
\begin{itemize}
    \item $f_1: \mathbb{R} \rightarrow \mathbb{R}: x \mapsto x-1$
    \item $f_2: \mathbb{R} \rightarrow \mathbb{R}: x \mapsto x^2$
    \item $f_3: \mathbb{R}_{>0} \rightarrow \mathbb{R}: x \mapsto x^2$
\end{itemize}

Dabei gilt $f_2 \neq f_3$

\textbf{Beispiele, die keine Funktionen sind:}
\begin{itemize}
    \item $f_4: \mathbb{R} \rightarrow \mathbb{R}: x \mapsto \frac{1}{x}$
    \item \begin{equation*}f_5: \mathbb{R} \rightarrow \mathbb{R}: x \mapsto \begin{cases} x+1 &\mbox{if } x \leq 0 \\ 
                  x & \mbox{if } x \geq 0 \end{cases}\end{equation*}
\end{itemize}



% subsection abbildungen (end)

% section vorlesung_23_10_2015 (end)

\newpage
\section{Vorlesung 28.10.2015} % (fold)
\label{sec:vorlesung_28_10_2015}

$Graph(f) = \{(x,f(x)) | x \in M\} \subset M \times N$ für $f: M \rightarrow N$

\textbf{Bemerkung:} Ist $\Gamma \subset M \times N$ so dass $\forall x \in M: \exists! y \in N: (x,y) \in \Gamma $, dann $\exists f: M \rightarrow N : \Gamma = Graph(f)$

\textbf{Definition:} Sei $f: M \rightarrow N$ eine Abbildung. $f$ heißt \underline{injektiv} genau wenn $\forall x,x' \in M: f(x) = f(x') \Rightarrow x = x'$ was genau heißt: Es gibt keine zwei verschiedenen Elemente in M, die dasselbe Bild in N haben, oder jedes y hat höchstens ein urbild.

\textbf{Definition:} $f$ heißt \underline{surjektiv}, genau wenn $\forall y \in N: \exists x \in M: f(x) = y$ was genau heißt $Im(f) = N $ und auch heißt jedes $y \in N $ hat mindestens ein Urbild

\textbf{Definition:} f heißt \underline{bijektiv} genau wenn f ist injektiv und surjektiv was heißt jedes $y \in N$ hat genau ein Urbild.

% Beispiele für inj sur un bij

\textbf{Beispiel:} $f_2: \mathbb{R} \rightarrow \mathbb{R}: x \mapsto x^2$ ist nicht surjektiv und nicht injektiv

Verändern wir $f_2: \mathbb{R}_{\geq0} \rightarrow \mathbb{R}_{\geq0}$ ist nun injektiv aber nicht surjektiv

$id_M : M \rightarrow M : x \mapsto x$ ist bijektiv

Ist $f: M \rightarrow N$ injektiv, do ist die Abbildung $M \rightarrow Im(f): x \mapsto f(x)$ bijektiv.

\textbf{Definition} Seien $f: M \rightarrow N, g: N \rightarrow P$ Abbildungen:

$g \circ f : M \rightarrow P: x \mapsto g(f(x))$ heißt die \underline{Komposition} oder \underline{Verkettung} von $f$ und $g$.

$M \xrightarrow{f} N \xrightarrow{g} P$

\textbf{Bemerkung:} Reihenfolge ist wichtig (verschiedene Definitions und Zielbereiche). Selbst, wenn Definitions und Zielbereich übereinstimmen.

$f: \mathbb{R} \rightarrow \mathbb{R} : x \mapsto x^2, g: \mathbb{R} \rightarrow \mathbb{R}: x \mapsto x+1$

$g \circ f: \mathbb{R} \rightarrow \mathbb{R}: x \mapsto x^2 \mapsto x^2+1$

$f \circ g: \mathbb{R} \rightarrow \mathbb{R}: c \mapsto x+1 \mapsto (x+1)^2$

Da $g \circ f(1)=2 \neq 4 = f \circ g(1)$ ist $g \circ f \geq f \circ g$

\textbf{Lemma:} Assoziativität der Komposition 

Seien $f: M \rightarrow N, g: N \rightarrow P, h: P \rightarrow Q$ Abbildungen. Dann gilt $(h \circ g) \circ f = h \circ (g \circ f)$. Wir schreiben deshalb auch $h \circ g \circ f$.

\textbf{Beweis:} 

Der Definitionsbereich von $(h \circ g) \circ f$ ist M.

Der Definitionsbereich von $h \circ (g \circ f)$ ist M.


Der Zielbereich von $(h \circ g) \circ f$ ist Q.

Der Zielbereich von $h \circ (g \circ f)$ ist Q.


Also stimmen Definitionsbereich und Zielbereich beider Funktionen überein.

Wir müssen noch die Abbildungsvorschriften überprüfen: Dazu sei $x \in M$ beliebig. Dann gilt $(h \circ g) \circ f) (x) = (h \circ g) \circ  f(x) = h(g(f(x))) = h(g \circ f(x)) = (h \circ (g \circ f)) (x)$. 

\textbf{Satz:} Sei $f: M \rightarrow N$ eine Abbildung.

\begin{enumerate}[label=\alph*)]
    \item $f$ ist genau dann bijektiv, wenn eine Abbildung $g: N \rightarrow M$ existiert mit $g \circ f = id_M$ und $f \circ g = id_N$
    \item Die Abbildung g ist dann eindeutig bestimmt und bijektiv. Sie heißt Inverse oder Umkehrabbildung von $f$ und wird mit $f^{-1} bezeichnet$.  
\end{enumerate}

\textbf{Beweis:}
\begin{enumerate}[label=\alph*)]
    \item \begin{itemize}
        \item [$\Leftarrow:$]

        Wir zeigen zunächst, dass $f$ surjektiv ist. Sei dazu $y \in N$. Setze $x := g(y) \in M$. Dann gilt $f(x) = f(g(y)) = (f \circ g) (y) = id_N (y) = y$.

        Wir zeigen: f ist injektiv. Seien dazu $x, x' \in M$ mit $f(x) = f(x')$. Dann gilt $x = id_m (x) = g \circ f (x) = g(f(x)) = g(f(x')) = g \circ f (x') = id_M (x') = x'$

        \item [$\Rightarrow:$]

        Da f bijektiv ist, existiert für jedes $y \in N$ genau ein urbilg $x_y \in M$ con y unter f, i.e. $f(x_y)=y$. Wir definieren: $g:N \rightarrow M: y \mapsto x_y$

        Dann gilt für $y \in N f \circ g y) = f(g(y)) = f(x_y) = y = id_N(y) $. Da $ f \circ g: N \rightarrow N, id_N: N \rightarrow N,$ forlgt $f \circ g = id_N$

        Sei $x \in M$ und $y := f(x) \in N$, dann ist $f(x_y) = y = f(x) \xrightarrow{f injektiv} x = x_y$

        Also ist für $x \in M g \circ f (x) = g(f(x)) = g(y) = x_y = x = id_M (x)$

        Da $g \circ f: M \rightarrow M, id_M: M \rightarrow M$, folgt $g \circ f = id_M$
    \end{itemize}
    \item Sei $h: N \rightarrow M$ eine Abbildung mit $h \circ f = id_M, f \circ h = id_N$.

    Zu zeigen ist $g = h$

    Für $y \in N$ gilt $f(g(y)) = (f \circ g) (y) = id_N (y) = y = f \circ h (y) = f(h(y))$

    Da f injektiv ist, folgt $g(y) = h(y)$.

    Damit ist die Eindeutigkeit von g gezeigt.

\end{enumerate}

g erfüllt die Vorraussetzungen in a) und ist daher bijektiv [a) $\Leftarrow$]

\textbf{Beispiel:} 

$f: \mathbb{R} \rightarrow \mathbb{R}: x \mapsto 2x +1$ ist bijektiv mit 

$f^{-1}: \mathbb{R} \rightarrow \mathbb{R}: y \mapsto \frac{1}{2}y-\frac{1}{2}$ denn

$(f \circ f^{-1}) (y) = 2( \frac{1}{2}y -\frac{1}{2}) + 1 = y = id_\mathbb{R}(y)$

$(f^{-1} \circ f) (x) = \frac{1}{2}(2x+1)-\frac{1}{2} = x = id_\mathbb{R} (x)$

\textbf{Achtung:} Die Notation $f^{-1}$ kann verwirren, denn Urbilder werden auch so bezeichnet. für bijektive Abbildung f ist $f^{-1}(y)=x$ (Umkehrabbildung) und $f^{-1}(\{y\})=\{x\}$ (Urbild).

\newpage

\subsection{Mächtigkeit von Mengen} % (fold)
\label{sub:m_chtigkeit_von_mengen}

\textbf{Definition:} Zwei Mengen $M$ und $N$ heißen \underline{gleich mächtig}, wenn es eine bijektive Abbildung $f: M \rightarrow N$ gibt.

\textbf{Lemma:} Seien $M$ und $N$ endliche Mengen. Dann gilt: $M$ und $N$ sind gleich mächtig genau dann, wenn sie gleich viele Elemente besitzen.

\textbf{Beweis:} 
\begin{itemize}
    \item [$\Rightarrow:$] Es existier $f: M \rightarrow N$ bijektiv. M ist endlich, schreibe $M = \{x_1,...,x_n\}, n \in \mathbb{N}$. 

    Betrachte de Bilder von $y_i := f(x_i)$

    Da f surjektiv ist, ist $N = \{y_1,...,y_n\}$.

    Da f injektiv ist, ist $y_i \neq y_j$ für $i \neq j$.

    Also hat $N$ wie $M$ $n$ Elemente

    \item [$\Leftarrow:$] Wir numerieren die Elemenre von M und N:

    $M = \{x_1,...,x_n\}$

    $N = \{y_1,...,y_n\}$

    Setze $f: M \rightarrow N: x_i \mapsto y_i$. 
\end{itemize}

f ist bijektiv.



% subsection m_chtigkeit_von_mengen (end)

% section vorlesung_28_10_2015 (end)

\newpage

\section{Vorlesung 30.10.2015} % (fold)
\label{sec:vorlesung_30_10_2015}

\textbf{Definition:} Für endliche Mengen nennen wir $|M|$ oder $\#M$ die \underline{Mächtigkeit} von M definiert als die Zahl der Elemente in M.

Wir nennen eine Menge \underline{abzählbar unendlich}, wenn sie gleich mächtig zu $\mathbb{N}$ ist, es also eine Bijektion zu $\mathbb{N}$ gibt.

Eine Menge heißt \underline{überabzählbar}, wenn sie weder endlich, noch abzählbar ist.

\textbf{Beispiel:} $\mathbb{Z}$ ist abzählbar unendlich. Hier ist eine bijektive Abbildung $\mathbb{N} \rightarrow \mathbb{Z}$:

\begin{equation*}
f: x \mapsto \begin{cases} 0 &\mbox{if } n = 0 \\ 
                           \frac{n+1}{2} & \mbox{if } n ungerade\\
                           -\frac{n}{2} & \mbox{if } n gerade \end{cases}
\end{equation*}

\textbf{Proposition:} \underline{Cantorsches Diagonalverfahren}

$\mathbb{Q}$ ist abzählbar unendlich.

\textbf{Beweis:} Wir listen die rationalen Zahlen auf:
\begin{equation*}
\begin{pmatrix}
    1 & \frac{1}{2} & \cdots & \frac{1}{n} \\
    -1 & \frac{-1}{2} & \cdots & \frac{-1}{n} \\
    2 & \frac{2}{2} & \cdots & \frac{2}{n} \\
    \vdots  & \vdots  & \ddots & \vdots  \\
    m & \frac{m}{2} & \cdots & \frac{m}{n} 
\end{pmatrix}
\end{equation*}

Und laufen an Hand der Diagonale durch die Matrik und sammeln jede Zahl einmal auf.

\textbf{Proposition:} Die Menge $\mathbb{R}$ der reellen zahlen ist überabzählbar.

\textbf{Beweis:} $\mathbb{R}$ ist nicht endlich. Wäre $\mathbb{R}$ abzählbar unendlich, so gäbe es eine bijektive Abbildung $\varphi: \mathbb{N} \rightarrow \mathbb{R}$.

Wir schreiben $\varphi(i)$ in Dezimaldarstellung $\forall i \in \mathbb{N}$:

$\varphi(0) = a_{0-p_0}\ a_{0-p_0+1} \cdots a_{00}\ a_{01} \cdots$

$\varphi(1) = a_{1-p_0}\ a_{1-p_0+1} \cdots a_{10}\ a_{11} \cdots$

Setze $a:= a_{00}\ a_{11}\ a_{22}\ \cdots$

Nun ändern wir jede Ziffer von a ab (Zum Beispiel $b_{ii}=1$ falls $a_{ii}=0$ $b_{ii}=0$ sonst) und erhalten $b:= b_{00}\ b_{11}\ b_{22}\ \cdots \in \mathbb{R}$ mit $a_{ii} \neq b_{ii} \forall i \in \mathbb{N}$

Da $\varphi$ bijektiv ist existiert ein $i \in \mathbb{N}$ mit $\varphi(i) = b$, also ist $a_{ii}=b_{ii}$, was ein Widerspruch ist.

\newpage

\textbf{Satz:} Sei $f: M \rightarrow N$ eine Abbildung zwischen endlichen Mengen mit $|M|=|N|$.

Dann sind folgende Aussagen äquivalent:

\begin{itemize}
    \item f ist injektiv
    \item f ist surjektiv
    \item f ist bijektiv
\end{itemize}
 
\textbf{Beweis:} $2 \Rightarrow 3$

Sei f surjektiv. 

Dann ist $|N|=\displaystyle\sum_{b \in N} |\{b\}| \leq \displaystyle\sum_{b \in N} |f^{-1}(b)| = |M|$

Da $|M| = |N|$ folgt $|\{b\}| = |f^{-1}(b)|=1 $ für alle b in N

Somit folgt: f ist bijektiv

$1 \Rightarrow 3$:

f ist injektiv.

Dann ist $|N| = \displaystyle\sum_{b \in N} |\{b\}| \geq \displaystyle\sum_{b \in N} |f^{-1}(b)| = |M|$

Wegen $|N| = |M|$ wieder $|\{b\}| = |f^{-1}(b)| = 1$ für alle b in N

Somit folgt: f ist bijektiv

Die Umkerung folgt per Definition.

\textbf{Korollar:} M,N endlich:

$f: M \rightarrow N$ injektiv $\Rightarrow |M| \leq |N|$

$f: M \rightarrow N$ surjektiv $\Rightarrow |M| \geq |N|$

folgt unmittelbar aus dem vorhergehenden Beweis.

\textbf{Korollar:} \underline{Schubfachprinzip}

Seien M,N endliche Mengen:

Sei $f: M \rightarrow N, |M| > |N|$, dann ist f nicht injektiv.

\textbf{Intuition:} N ist eine Menge von Schubladen. Packen wir in jede Schublade ein Element von M, dann bleibt mindestens eins übrig, was in eine Schublade muss. Damit hat eine Schublade zwei Elemente.

\textbf{Proposition:} Unter beliebigen $n^2+1$ Punkten $P_1,\cdots,P_{n^2+1}$ in einem Quadrat der Kantenlänge n gibt es zwei Punkte mit Abstand $\leq \sqrt{2}$

\textbf{Beweis:}

Zerlege das Quadrat in $n^2$ Quadrate $Q_i$ der Kantenlänge 1.

Wir definieren $f: \{P_1,\cdots,P_{n^2+1}\} \rightarrow \{Q_i\}$ mit $f(P_j) = Q_i$ falls $P_j \in Q_i$.

Per Schubfachprinzip existiert ein Quadrat, was zwei Punkte enthält. Diese beiden Punkte haben Abstand kleine $\sqrt{2}$

\newpage

\textbf{Definition:}

Sei M eine Menge. Die Menge $\mathcal{P}(M) := \{A | A \subset M\}$ aller Teilmengen von M heißt die \underline{Potenzmenge} von M.

\textbf{Beispiel:} 

$\mathcal{P}(\emptyset) = \{\emptyset\}$

$\mathcal{P}(\{1,3\}) = \{\emptyset,\{1\},\{2,\},\{1,2\}\},$

\textbf{Satz:} $\mathcal{P}(M)$ ist nicht gleichmäßig zu M

\textbf{Beweis:}

Sei $M = \emptyset$, dann ist $|M|=0$, aber $|\mathcal{P}(M)|=1$, $1\neq 0$

Sei $M \neq \emptyset$. Angenommen, es gäbe $f: M \rightarrow \mathcal{P}(M)$ bijektiv

$\forall x \in M$ ist $f(x) \in \mathcal{P}(M)$ also $f(x) \subset M$

Betrachte $U = \{x \in M | x \notin f(x)\}, U \subset M$ also $U \in \mathcal{P}(M)$

f ist surjektiv, also existiert ein $u \in M: U = f(u)$

Angenommen $u \notin U = f(u) \Rightarrow u \in U$

Angenommen $u \in U = f(u) \Rightarrow u \notin U$

Das ist ein Widerspruch.

\subsection{Relationen} % (fold)
\label{sub:relationen}

\textbf{Definition:} Seien M,N Mengen, jede Teilmenge $R \subset M \times N$ heißt \underline{Relation} zwischen M und N.

Ist R eine Relation zwischen M und N, $x \in M, y \in N$, dann sagen wir x steht in Relation zu y bezüglich R, wenn gilt, dass $(x,y) \in R$.

\textbf{Beispiel:} Der Graph einer Funktion ist eine Relation, bei der jedes $x \in M$ zu genau einem $y \in N$ in Relation steht.

$M = N = \mathbb{R}, R = \{(x,y) \in \mathbb{R}^2 | x \leq y\}$ ist eine Relation.

\textbf{Definiton:} Sei M eine Menge. Eine \underline{Ordnungsrelation} (Halbordnung/partielle Ordnung) auf M ist eine Relation $R \subset M \times M$ so dass $\forall x,y,z \in M:$

\begin{itemize}
     \item $(x,x) \in R$ \underline{Reflexivität}
     \item $(x,y) \in R, (y,x \in R \Rightarrow x=y$ \underline{Antisymmetrie}
     \item $(x,y) \in R, (y,z) in R \Rightarrow (x,z) \in R$ \underline{Transitivität}
 \end{itemize} 

 \textbf{Beispiel:} $M = \mathbb{R}, R = \{(x,y) | x \leq y\}$ ist Ordnungsrelation

 \textbf{Notation:} Wir schreiben $x \leq y \Leftrightarrow (x,y) \in R$ und sprechen von der Ordnungsrelation R. Wir sprechen auch von der teilgeordneten/partiell geordneten Menge $(M,\leq)$

 \textbf{Beispiel:} Sei M eine Menge, $\mathcal{P}(M)$ ihre Potenzmenge

 Sei $R \subset \mathcal{P}(M) \times \mathcal{P}(M)$ definiert durch $R = \{(A,B) | A \subset B\}$ eine Ordnungsrelation.

% subsection relationen (end)

% section vorlesung_30_10_2015 (end)

\section{Vorlesung 4.11.2015} % (fold)
\label{sec:vorlesung_4_11_2015}

\textbf{Definiton}

Sei M eine Menge. 

\begin{itemize}
	\item Eine Ordnungsrelation heißt \underline{Totalordnung}, falls je 2 Elemente vergleichbar sind. Da heißt, dass sie zu einander in einer Relation stehen. Also $\forall x,y \in M : x \leq y \lor y \leq x$
	\item Ist $\leq$ eine Ordnungsrelation auf M, $A \subset M, x \in A$, so heißt x minimal (beziehungsweise maximal) in A, falls gilt 
	$\forall y \in A$ gilt, mit $y \leq x$ (beziehungsweise $x \leq y$) folgt $y = x$.
	\item Eine Totalordnung heißt \underline{Wohlordnung}, falls jede nicht-leere Teilmenge von M ein minimales Element besitzt. 
	Das minimale beziehungsweise maximale Element ist eindeutig, sofern es existiert. Wir bezeichnen es as $Min(A)$ oder $Max(A)$.
\end{itemize}

\textbf{Beispiel:}
\begin{itemize}
	\item $(\mathbb{R}, \leq )$ ist totalgeordnet, aber nicht wohlgeordnet
	\item $(\mathbb{Z}, \leq )$ ist totalgeordnet, aber nicht wohlgeordnet
	\item $(\mathbb{N}, \leq )$ ist wohlgeordnet
	\item $\mathbb{Z}$ mit der Anordnung $0 < -1 < 1 < -2 < 2 \cdots$ ist wohlgeordnet
\end{itemize}

\textbf{Definition:}

Sei M eine Menge. Eine \underline{Äquivalenzrelation} auf M ist eine Relation, so dass $\forall x,y,z \in M$ gelten:
\begin{itemize}
	\item Reflexivität $(x,x) \in R$ oder $x \sim x$
	\item Symmetrie $(x,y) \in R \Rightarrow (y,x) \in R$ oder $x \sim y \Rightarrow y \sim x$
	\item Transitivität $(x,y),(y,z) \in R \Rightarrow (x,z) \in R$ oder $x \sim y, y \sim z \Rightarrow x \sim z$
\end{itemize}

\textbf{Notation:}

Sei M eine Menge, R eine Äquivalenzrelation, wir schreiben $x \sim y$ statt $(x,y) \in R$

\textbf{Definition:}

Sei M eine Menge und $\sim$ eine Äquivalenzrelation.

Für $ x \in M$ heißt $[x] := \{y \in M | y \sim x\}$ die Äquivalenzklasse von x. Jedes $y \in [x]$ heißt Vertreter oder Repräsentant der Klasse $[x]$.

\textbf{Notation:}

$M /_\sim := \{[x] | x \in M \}$ bezeichnet die Menge der Äquivalenzklassen.

\textbf{Beispiel:}
\begin{itemize}
	\item Sei $M$ beliebige Menge, $=$ ist eine Äquivalenzrelation, jede Äquivalenzklasse besteht aus einem Element.
	\item $M = \{Studenten\}, x \sim y: \Leftrightarrow x$ und $y$ studieren das selbe Fach
	\item Rationale Zahlen, $\frac{1}{2}$ und $\frac{2}{4}$ wollen wir nicht unterscheiden.
	Definiere auf $\mathbb{Z} \times \mathbb{Z}$ eine Äquivalenzrelation $\sim$ durch 
	$(p,q) \sim (p',q'):\Leftrightarrow p \cdot q' = p' \cdot q$.
	Die rationalen Zahlen sind die Menge der Äquivalenzklassen dieser Relation.
\end{itemize}

\textbf{Lemma:}

Sei M eine Menge und $\sim$ eine Äquivalenzrelation auf M. 

Die Äquivalenzklassen bilden eine disjunkte Zerlegung von M, anders gesagt, jedes Element von M liegt in genau einer Äquivalenzklasse. Für zwei Äquivalenzklassen $[x],[y]$ gilt entweder $[x]=[y]$ oder $[x] \cap [y] = \emptyset$.

\textbf{Beweis:}

Sei $x \in M$. Aus $x \sim x$ folgt $x \in [x] \subset \bigcup_{[y] \in M/_\sim}[y] \Rightarrow \bigcup_{[y] \in M/_\sim}[y] = M$.

Seien $[x],[y] \in M/_\sim $ mit $[x]\neq[y] $. Angenommen $[x] \cap [y] \neq \emptyset$, 

dann existiert $z \in [x]\cap[y] \Rightarrow z \sim x, z \sim y \Rightarrow x \sim z, z \sim y \Rightarrow x \sim y$

Sei $u \in [x] \Rightarrow u \sim x$, da auch $x \sim y \Rightarrow u \sim y \Rightarrow u \in [y] \Rightarrow [x] \subset [y]$.

Genauso umgekehrt. Also ist $[x] \subset [y] \land [y] \subset [x] \rightarrow [x] = [y]$ was ein Widerspruch für die Annahme $[x] \neq [y]$ ist.

Also sind $[x]$ und $[y]$ verschiedene Äquivalenzklassen und disjunkt.

\subsection{Rechnen mit Restklassen} % (fold)
\label{sub:rechnen_mit_restklassen}

\textbf{Definition:}

$b \in \mathbb{Z}$ heißt Teiler von $a \in \mathbb{Z}$, falls $c \in \mathbb{Z}$ existiert mit $b\cdot c = a$. Wir scheiben $b | a$.

\textbf{Satz:} \underline{Division mit Rest in $\mathbb{Z}$}

Sei $a \in \mathbb{Z}$ und $m \in \mathbb{N}$. Dann gibt es eindeutig bestimmte Zahlen $q \in \mathbb{Z}$ und $r \in \{0,1,\cdots,m-1\}$ mit $a=q\cdot m + r$. a heißt Quotient, r heißt Rest von a bei Division durch m.

\textbf{Definition:}

Zwei ganze zahlen $a,b$ heißen \underline{kongruent modulo m} $(m \in \mathbb{N})$ wenn $m | a-b$. Wir schreiben $a \equiv b\mod\ m$.

\textbf{Lemma:} $\equiv \mod m$ ist eine Äquivalenzrelation.

\textbf{Beweis:}

$m | 0$, denn $0 \cdot m = 0 \Rightarrow m | a-a \forall a \in \mathbb{Z} \Rightarrow a \equiv a \mod m \Rightarrow \equiv $ ist reflexiv.

Sei $a \equiv b \mod m \Rightarrow m | a-b \Rightarrow \exists c: m \cdot c = a-b \Rightarrow m \cdot (-c) = b -a \Rightarrow m | b-a \rightarrow b \equiv a \mod m \Rightarrow$ Symmetrie

Sei $a \equiv b, b \equiv c \mod m \Rightarrow m | a-b, m | b-c \Rightarrow \exists c,c': m \cdot c  = a-b, m \cdot c' = b-a \Rightarrow m \cdot (c + c') = a-c \Rightarrow m | a-c \rightarrow a \equiv c \mod m \Rightarrow $ Transitivität

\textbf{Bemerkung:} Man kann Kongruenz mod m auch anders definieren: $a \equiv b \mod m \Leftrightarrow a$ und $b$ haben bei Division durch $m$ den selben Rest. 

Es gilt $a = q_1 m + r_1, b = q_2 m + r_2, q_i \in \mathbb{Z}, r_i \in \{0,\cdots,m-1\}$

Also gilt $a-b = (q_1 - q_2) \cdot m + (r_1 - r_2)$

Also gilt $a \equiv b \mod m \Leftrightarrow r_1 -r_2 = 0 \Leftrightarrow r_1=r_2$

Die Äquivalenzklassen der Relation $\equiv$ nennen wir Restklassen. In jeder Restklase gibt es einen besonderen Vertreter, den Rest $r \in \{0,\cdots,m-1\}$.

$\mathbb{Z}/_{\equiv \mod m} =: \mathbb{Z}/_{m \mathbb{Z}} =: \mathbb{Z}_m = \{[0],[1],\cdots,[m-1]\}$

\textbf{Definition:} Auf $\mathbb{Z}_m$ definieren wir eine Addition und Multiplikation durch $[a] + [b] := [a+b]$ und $[a] \cdot [b] := [a \cdot b]$

\textbf{Achtung:} \underline{Wohldefiniertheit}:

Frage: Für $[a'] = [a]$, ist dann auch $[a'+b] = [a+b]$?

Hängt die Definition vom gewählten Vertreter ab? Nur wenn dies nicht der Fall ist, dann ist das eine wohldefinierte Definition.

\textbf{Beweis:}

Sei $a' \in [a] \Rightarrow a \equiv a' \mod m \Rightarrow a - a' = c \cdot m$ für $c \in \mathbb{Z}$

Also $(a+b) - (a'+b) = c \cdot m \Rightarrow a+b \equiv a'+b \mod m \Rightarrow [a+b] = [a'+b]$

Genauso $ab-a'b = (a-a')b = c \cdot m \cdot b \Rightarrow [ab]=[a'b]$

Folgende Rechenregeln gelten (wie man leicht sieht):

\begin{itemize}
	\item Kommutativität von +: $[a]+[b]=[b]+[a]$
	\item Kommutativität von $\cdot$: $[a]\cdot[b]=[b]\cdot[a]$
	\item Additatives neutrales Element: $[a]+[0]=[a]$
	\item Mutiplikatives neutrales Element: $[a]\cdot[1]=[a]$
	\item Additive inverse Elemente: $[a]+[-a]=[0]$
	\item Assoziativität der Addition: $([a]+[b])+[c]=[a]+([b]+[c])$
	\item Assoziativität der Multiplikation: $([a]\cdot[b])\cdot[c]=[a]\cdot([b]\cdot[c])$
	\item Distributivität: $[a] \cdot ([b]+[c])=[a]\cdot[b]+[a]\cdot[c]$
\end{itemize}

\textbf{Definition:}

Seien $a,b \in \mathbb{Z} \setminus \{0\}$

Wir nennen $c \in \mathbb{N}\setminus \{0\}$ größten gemeinsamen Teiler oder $ggT(a,b) \cdot d$, wenn:

\begin{itemize}
	\item $d | a$
	\item $d | b$
	\item $\forall c \in \mathbb{Z}: c | a, c | b \Rightarrow c | d$
\end{itemize}

\textbf{Bemerkung:} Es gibt höchstens einen $ggT(a,b)$

Angenommen $d,d'$ erfüllen alle drei Punkte, dann liefert der dritte Punkt angewendet auf d $d | d'$ und umgekehrt $d' | d$

Somit $\exists c,c' : cd=d', c'd'=d \Rightarrow c\cdot c' \cdot d' = d' \Rightarrow c \cdot c' = 1$. Da $d,d' > 0$ folgt $c = c' = 1$ also $d = d'$.

% subsection rechnen_mit_restklassen (end)

% section vorlesung_4_11_2015 (end)

\end{document}
 
